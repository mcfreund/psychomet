\documentclass[]{article}

\usepackage{amsmath}

%opening
\title{comparing univariate and multivariate techniques with euclidean distance}
\author{mike freund}

\begin{document}

\maketitle


\section*{notation}

For a given subject, task, and parcel, let $x_v$ be the vector of ``raw data" (i.e., the contrast estimates obtained from the GLM: e.g., Hi\textunderscore Lo in Stroop).
I'll call this vector the ``activation pattern" (a pattern across vertices).
The subscript $v$ indexes the vertex within the parcel, where $v \text{ in } 1, \dots, V$ vertices.
I'll indicate which subject the data are from by the superscript $x_v^{(s)}$, for $s \text{ in } 1, \dots, S$ subjects.
E.g., for subject $s = 1$, the activation pattern is indicated by $x_v^{(1)}$.

\section*{euclidean distance: multivariate case}

The distance between subject 1 and 2's activation patterns is given by

\begin{equation}
	\textit{multivariate d}(1, 2) = \sqrt{\sum_{v = 1}^{V}(x_v^{(1)} - x_v^{(2)})^2}
\end{equation}


But, we will want to use this formula instead:

\begin{equation}
	\textit{multivariate d}_{\text{scaled}}(1, 2) = 
	\sqrt{
		\frac
		{\sum_{v = 1}^{V}(x_v^{(1)} - x_v^{(2)})^2}
		{V}
	}
\end{equation}

This is because the magnitude of $d^2$ will scale with the number of dimensions $V$.
Our main goal is to compare the sensitivity of $d$ when all dimensions are used, to when only one is used (the mean).
Dividing the distances by $\sqrt{V}$ puts these metrics on the same scale.

\section*{euclidean distance: univariate case}

We will use the same formula for the univariate case.
However, now, we will not compute the distances between the pattern vectors $x_v^{(s)}$, but between the \textit{means} of the pattern vectors.

So, first, calculate the across-vertex mean of each pattern vector
\[
\bar{x}^{(1)} = \frac{1}{V} \sum_{v = 1}^{V}{x}_v^{(1)}
\]

and 
\[
\bar{x}^{(2)} = \frac{1}{V} \sum_{v = 1}^{V}{x}_v^{(2)}
\]

Then simply plugging these guys into Equation 2, we get

\[
\textit{univariate d}_{\text{scaled}}(1, 2) = \sqrt{\frac{\sum_{v = 1}^{V}(\bar{x}^{(1)} - \bar{x}^{(2)})^2}{V}}
\]

But because ``$V$'', in this case, actually equals 1 (the mean is a scalar value), we can remove the denominator and summation operation

\[
\textit{univariate d}_{\text{scaled}}(1, 2) = \sqrt{(\bar{x}^{(1)} - \bar{x}^{(2)})^2}
\]

which is equal to the absolute value of the difference between means:

\begin{equation}
	\textit{univariate d}_{\text{scaled}}(1, 2) = |\bar{x}^{(1)} - \bar{x}^{(2)}|
\end{equation}


\section*{inferential statistics to perform}


There are three things we want to know:
\begin{enumerate}
	\item Perform the within--between subject `fingerprinting' procedure using the \textit{univariate d} (Equation 3). I.e., contrast the average within-subject, between-run \textit{univariate d} with the average \textit{between}-subject, between-run \textit{univariate d}.
	In which parcels is this contrast significantly \textit{less than zero}? (I.e., smaller within-subject distance than between-subject distance)?
	\item Perform the same `fingerprinting' procedure using the \textit{multivariate d} (Equation 2); which parcels are identified as being significant?
	\item Directly compare the magnitude of the contrast statistic between \textit{univariate} and \textit{multivariate} measures.
	This could be performed via a paired-sample t-test.
\end{enumerate}




\end{document}
